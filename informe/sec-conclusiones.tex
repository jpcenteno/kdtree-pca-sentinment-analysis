\section{Conclusiones}%

Para el método de la potencia determinamos un criterio de corte buen rendimiento asociado a un $\epsilon$ que no costara mucho en términos de tiempo, sobre todo considerando que al aplicar deflación este método se llama continuamente.

De mismo modo usamos una estructura de datos alternativa para que $kNN$ fuera mas perfomante y optimizamos parámetros de $k$ y $\alpha$ para tal método y PCA.

Potencial trabajo futuro sería experimentar sobre las normas que se usan para buscar vecinos cercanos y modos de vectorización que optimicen la información contenida en los vectores. 
\label{sec:conclusiones}

\section{Conclusiones}%

Para el método de la potencia determinamos un criterio de corte con buen rendimiento basado en diferencias residuales e incrementales con una tolerancia $\epsilon = 10^{-6}$ que no costara mucho en términos de tiempo, sobre todo considerando que al aplicar deflación este método se llama continuamente para usar PCA.

Analizamos la precisión respecto del tamaño del set de entrenamiento al que se somete kNN con PCA notando que progresan de manera proporcional.

Vimos que sobre el set de entrenamiento y testeo de imdb\_small provisto por la cátedra no fue posible conseguir accuracy scores por encima de $0.68$ para kNN en una sola iteración de optimización de parámetros independientemente del agregado de PCA. El $k$ que optimiza tal situación es $k=1826$ obteniendo accuracy $0.685$.

La experimentación sobre PCA no fue enteramente inconclusiva: pudimos ver que la función objetivo en función de $k$ acompaña (con una dimensión más) a aquella de la experimentación de kNN sin PCA sugiriendo que las mejoras que se puedan hacer en términos de vectorización sobre una alternativa impactarían de manera similar en la otra.

Los métodos de experimentación y desarrollo planteados son la base sobre la cual se podría reiterar la optimización de parámetros de modo de conseguir progresivamente mejores resultados.
Potencial trabajo futuro sería experimentar sobre las normas que se usan para buscar vecinos cercanos y modos de vectorización que optimicen la información contenida en los vectores. Se podría extender el análisis de recorte según frecuencias de la vectorización a kNN con PCA (a pesar de que nuestro mejor kNN con PCA no resultó ser más preciso que nuestro mejor kNN sin PCA) para buscar mejoras también ahí.

\label{sec:conclusiones}

\section{Resumen}%
\label{sec:abstracto}

% ¿Cual es el problema estudiado en el informe?
Distintos sitios de reseñas cinematográficas permiten a los usuarios escribir
sus propias criticas y dar una puntuación a las películas.
% ¿Por que nos interesa este problema?
Al combinar texto con una clasificación numérica, se conforma un conjunto de
datos interesante para llevar a cabo análisis supervisado de sentimiento sobre
el lenguaje de los usuarios.

% ¿Que tecnicas probamos?
% 1. kNN sobre Bag of Words:
En este informe se evalúa el desempeño de clasificadores basados en \knn{}
sobre \textit{bag of words} de reseñas de películas para clasificar el
sentimiento subyacente en el texto.
% 2. kNN con PCA
Adicionalmente se evaluó el desempeño de utilizar PCA en el pre
procesamiento del texto.
Para implementar PCA tuvimos que hacer nuestro propio método de la potencia con deflación.

% Adelanto de las conclusiones
Todos estos métodos requieren parámetros que afectan fuertemente el resultado en
términos de precisión final como de latencia. Nos interesa encontrar aquellos
que minimicen sacrificios en tal \textit{tradeoff}.

La experimentación extensa sobre $\alpha$ y \emph{k}, no arrojó tan
buenos resultados como una adecuada vectorización del vocabulario. La
elección del punto de corte en palabras de más y menos frecuencia tuvo
tanto o más peso que la elección de la cantidad de vecinos y
componenetes principales.

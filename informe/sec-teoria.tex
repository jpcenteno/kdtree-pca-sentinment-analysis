\section{Introducción Teórica}%
\label{sec:introduccion_teorica}

\subsection{Análisis de Sentimiento}%
\label{sub:analisis_de_sentimiento}

\subsection{Bag of Words}%
\label{sub:bag_of_words}

\textit{Bag of Words} es un sistema de representación utilizado en
procesamiento del lenguaje natural. Cada documento pasa a ser representado por
su multiconjunto de palabras. Se pierde el orden original de las palabras en el
texto. Se utiliza en problemas donde el vocabulario del documento es suficiente
para estimar el sentimiento subyacente.

% FIXME tirar un ejemplo ilustrativo

\subsection{k Nearest Neighbors}%
\label{sub:k_nearest_neighbors}

% knn in a nutshell:
El algoritmo \textit{$k$ Nearest Neighbors}, \knn{} es un clasificador
supervisado que estima la categoría de una instancia basandose en su vecindad
con instancias ya conocidas.
% Para que tipos de problemas sirve
Se desempeña mejor en problemas donde la vecindad espacial entre
observaciones es un factor importante en la clasificación de instancias.

% Como funciona el algoritmo
Para una instancia de entrada $x$, se calcula su distancia con todo el set de
entrenamiento. La clasificación estimada resultante es la categoría modal (voto
mayoritario) entre los $k$ vecinos mas cercanos a $x$ en el set de
entrenamiento.

% Cuales son los parámetros
Influye en \knn{} la elección del $k$ en el rango $1 \leq k \leq N$, así como
la elección de la norma a utilizar para medir distancia.
% Que pasa con `k` alto.
Para un valor de $k$ alto, la estimación puede verse afectada por la categoría
modal en el conjunto de entrenamiento.
% Que pasa con `k` chico
Para una eleción de $k$ chico, la clasificación puede verse afectada por la
presencia de outliers \textit{outliers} en su cercanía.
% FIXME hablar de distintas normas.
FIXME hablar de distintas normas.

% FIXME hablar de algo que presente por que podemos necesitar PCA.

\subsection{Principal Component Analysis}%
\label{sub:principal_component_analysis}

\textit{Principal Component Analysis}, PCA, es un algoritmo de reducción de
dimensionalidad que transforma los datos de entrada en un espacio de
\textit{features} correlacionadas a un conjunto de valores cuyas
\textit{features} no guarden correlación entre sí.
